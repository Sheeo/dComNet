\documentclass[12pt,a4paper]{article}
\usepackage[utf8]{inputenc}
\usepackage{amsmath,amssymb}
\usepackage{listings}

\setlength{\parindent}{0in}
\setlength{\parskip}{0.1in}

\begin{document}
\title{dComNet Uge 1 Aflevering}
\author{Mathias Rav \\
		Michael Søndergaard \\
		DAT-3}
\date{November 2010}
\maketitle

\section{Multiplikation med imul-metoden}
Hvis vi kører programmet imul med parametrene $x=-1$ og $y=5$, får vi resultatet $0$

Som det kan ses fra stack-tracen:

IJVM Trace of imul.bc Thu Nov  4 20:57:23 2010

\begin{tabular}{llllllllll}
                  &           & stack = 0,& 1,& 16 \\
bipush 44         & [10 2c]   & stack = 44,& 0,& 1,& 16 \\
bipush -1         & [10 ff]   & stack = -1,& 44,& 0,& 1,& 16 \\
bipush 5          & [10 05]   & stack = 5,& -1,& 44,& 0,& 1,& 16 \\
invokevirtual 0   & [b6 00 00]& stack = 15,& 51,& 0,& 5,& -1,& 22,& 0,& 1 \\
bipush 0          & [10 00]   & stack = 0,& 15,& 51,& 0,& 5,& -1,& 22,& 0 \\
istore 3          & [36 03]   & stack = 15,& 51,& 0,& 5,& -1,& 22,& 0,& 1 \\
iload 1           & [15 01]   & stack = -1,& 15,& 51,& 0,& 5,& -1,& 22,& 0 \\
iflt 25           & [9b 00 19]& stack = 15,& 51,& 0,& 5,& -1,& 22,& 0,& 1 \\
iload 3           & [15 03]   & stack = 0,& 15,& 51,& 0,& 5,& -1,& 22,& 0 \\
ireturn           & [ac]      & stack = 0,& 0,& 1,& 16 \\
ireturn           & [ac]      & stack = 0 \\
\end{tabular}

return value: 0

\end{document}
