\documentclass[12pt,a4paper]{article}
\usepackage[utf8]{inputenc}
\usepackage[danish]{babel}
\usepackage{amsmath,amssymb}
\usepackage{listings}

\setlength{\parindent}{0in}
\setlength{\parskip}{0.1in}

\newcommand{\imul}{\texttt{imul}}

\begin{document}
\title{dComNet Uge 1 Aflevering}
\author{Mathias Rav \\
		Michael Søndergaard \\
		DAT-3}
\date{November 2010}
\maketitle

\section{Multiplikation med \imul{}-metoden, i C}
Hvis man kører \imul{}-metoden med parametrene $x=2$ og $y=-4$, får man $-8$.

Programmet vil altid multiplicere korrekt når $x$ er positiv, da addition/subtraktion er defineret for alle heltal.
Dog er programmets while-løkke ikke defineret så der kan itereres over $x$ i positiv retning.
Derfor vil programmet give 0, hvis $x$ er negativ.

\subsection{Induktionsbevis for, at resultatet altid er korrekt når x er positiv og y er negativ}

Om funktionen $f(x, y, p) = f(x-1, y, p+y)$, $f(0, y, p) = p$ antager vi, at $f(x, y, p) = xy+p$ gælder.
\paragraph{Basis} $x = 0$
$f(0, y, p) = p$, ved definitionen
\paragraph{Induktion på x}
Antag at $f(x-1, y, p) = (x-1)y+p$
så er $f(x, y, p) = f(x-1, y, p+y) = (x-1)y+p+y = xy+p$



\section{Multiplikation med \imul{}-programmet, i IJVM}
Hvis vi kører programmet \imul{} med parametrene $x=-1$ og $y=5$, får vi resultatet $0$

Det fremgår af stack-tracen:

IJVM Trace of \texttt{imul.bc} Thu Nov  4 20:57:23 2010

\begin{tabular}{llllllllll}
                &                    & stack = 0,& 1,& 16 \\
bipush 44       & \texttt{[10 2c]   }& stack = 44,& 0,& 1,& 16 \\
bipush -1       & \texttt{[10 ff]   }& stack = -1,& 44,& 0,& 1,& 16 \\
bipush 5        & \texttt{[10 05]   }& stack = 5,& -1,& 44,& 0,& 1,& 16 \\
invokevirtual 0 & \texttt{[b6 00 00]}& stack = 15,& 51,& 0,& 5,& -1,& 22,& 0,& 1 \\
bipush 0        & \texttt{[10 00]   }& stack = 0,& 15,& 51,& 0,& 5,& -1,& 22,& 0 \\
istore 3        & \texttt{[36 03]   }& stack = 15,& 51,& 0,& 5,& -1,& 22,& 0,& 1 \\
iload 1         & \texttt{[15 01]   }& stack = -1,& 15,& 51,& 0,& 5,& -1,& 22,& 0 \\
iflt 25         & \texttt{[9b 00 19]}& stack = 15,& 51,& 0,& 5,& -1,& 22,& 0,& 1 \\
iload 3         & \texttt{[15 03]   }& stack = 0,& 15,& 51,& 0,& 5,& -1,& 22,& 0 \\
ireturn         & \texttt{[ac]      }& stack = 0,& 0,& 1,& 16 \\
ireturn         & \texttt{[ac]      }& stack = 0
\end{tabular}

return value: $0$


Hvis vi kører programmet med parametrene $x=2$ og $y=3$, får vi resultatet $6$

Som det kan ses fra stack-tracen i filen "imul2.trace", så er stakket størst, efter instruktionen \texttt{invokevirtual 0 [b6 00 00]}. Den har en størrelse på $8$.

Stakkens størrelse er konstant for $-128 < x < 128$ og $-128 < y < 128$, da der ved tal større end 127, skal bruges flere registre for, at lagre x og y.

\end{document}
