\documentclass[12pt,a4paper]{article}
\usepackage[utf8]{inputenc}
\usepackage[danish]{babel}
\usepackage{amsmath,amssymb,amsthm}
\usepackage{listings}
\usepackage{color}
\lstloadlanguages{C,sh,JVMIS}
\setlength{\parindent}{0in}
\setlength{\parskip}{0.1in}

\newcommand{\ishl}{\texttt{ishl}}
\newcommand{\ishr}{\texttt{ishr}}
\newcommand{\iushr}{\texttt{iushr}}
\newcommand{\ijvm}{\textsc{ijvm}}
\newcommand{\ijvmmal}{\texttt{ijvm.mal}}
\newcommand{\ijvmspec}{\texttt{ijvm.spec}}

\newtheorem{theorem}{Sætning}

\lstset{ %
language=JVMIS,
basicstyle=\footnotesize,       % the size of the fonts that are used for the code
numbers=left,                   % where to put the line-numbers
numberstyle=\footnotesize,      % the size of the fonts that are used for the line-numbers
stepnumber=1,                   % the step between two line-numbers.
numbersep=5pt,                  % how far the line-numbers are from the code
backgroundcolor=\color{white},  % choose the background color. You must add \usepackage{color}
showspaces=false,               % show spaces adding particular underscores
showstringspaces=false,         % underline spaces within strings
showtabs=false,                 % show tabs within strings adding particular underscores
tabsize=2,	          	        % sets default tabsize to 2 spaces
captionpos=b,                   % sets the caption-position to bottom
breaklines=true,                % sets automatic line breaking
breakatwhitespace=false,        % sets if automatic breaks should only happen at whitespace
title=\lstname,                 % show the filename of files included with \lstinputlisting;
}


\title{Computers and Networks (Q2/2010)}
\author{Mathias Rav, 20103940 \\
		Michael Søndergaard, 20104223 \\
		DAT-3}
\date{Week 5, December 5, 2010}

\begin{document}
\maketitle

\section{Recursive Fibonacci in IA-32}
We have realised the Fibonacci algorithm \texttt{afib.s} in IA-32 symbolic machine language as such:

\lstinputlisting{afib.s}

We have commented the source code as to explain what happens where, but for reference:

Establish stack: line \texttt{77} \\
Transfer parameters and recurse: line \texttt{86} \\
Make space for local variables: line \texttt{13} (in \texttt{\_start} - no locals in \texttt{fib}) \\
Return values: lines \texttt{90} and \texttt{98} \\
Tear down stack: lines \texttt{102}-\texttt{103}

When run with $n=5$, the program returns the following output:
\lstset{numbers=none}
\begin{lstlisting}
$ ./afib 5  
5
\end{lstlisting}


In the following stack trace, the higher memory addresses are at the top, and
the stack grows downwards with decreasing memory addresses. The ``old eip''
memory cells contain addresses that point into the text section. The ``old
ebp'' memory cells contain addresses to memory cells above themselves and
realize the C calling conventions.


Stack trace at the following point:   \\
In afib.s:51, \_start calling fib(4), \\
   afib.s:87, fib(4) calling fib(3),  \\
   afib.s:87, fib(3) calling fib(2),  \\
   afib.s:87, fib(2) calling fib(1),  \\
   afib.s:102, fib(1) about to return 1 \\
\\
   \vdots
\\

\begin{tabular}{l r}
argv[2] &                                   \\
argv[1] &       2nd argument (n)            \\
argv[0] &       1st argument (program name) \\
argc    &       argument count              \\
old ebp &       ebp of \_start              \\
loc 1   &       endptr (for strtol)         \\
loc 2   &       n (passed to fib)           \\
4       &       1st argument to fib         \\
old eip &                                   \\
old ebp &       ebp of fib(4)               \\
3       &       1st argument to fib         \\
old eip &                                   \\
old ebp &       ebp of fib(3)               \\
2       &       1st argument to fib         \\
old eip &                                   \\
old ebp &       ebp of fib(2)               \\
1       &       1st argument to fib         \\
old eip &                                   \\
old ebp &       ebp of fib(1)               \\
\end{tabular}

The results are as expected.

\end{document}
