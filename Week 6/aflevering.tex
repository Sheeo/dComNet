\documentclass[12pt,a4paper]{article}
\usepackage[utf8]{inputenc}
\usepackage[danish]{babel}
\usepackage{amsmath,amssymb,amsthm}
\usepackage{listings}
\usepackage{color}
\usepackage{multicol}
\setlength{\parindent}{0in}
\setlength{\parskip}{0.1in}

\lstset{ %
basicstyle=\footnotesize,       % the size of the fonts that are used for the code
numbers=left,                   % where to put the line-numbers
numberstyle=\footnotesize,      % the size of the fonts that are used for the line-numbers
stepnumber=1,                   % the step between two line-numbers.
numbersep=5pt,                  % how far the line-numbers are from the code
backgroundcolor=\color{white},  % choose the background color. You must add \usepackage{color}
showspaces=false,               % show spaces adding particular underscores
showstringspaces=false,         % underline spaces within strings
showtabs=false,                 % show tabs within strings adding particular underscores
tabsize=2,	          	        % sets default tabsize to 2 spaces
captionpos=b,                   % sets the caption-position to bottom
breaklines=true,                % sets automatic line breaking
breakatwhitespace=false,        % sets if automatic breaks should only happen at whitespace
title=\lstname,                 % show the filename of files included with \lstinputlisting;
}


\title{Computers and Networks (Q2/2010)}
\author{Mathias Rav, 20103940 \\
        Michael Søndergaard, 20104223 \\
        DAT-3}
\date{Week 6, December 12, 2010}

\begin{document}
\maketitle

\section{UDP ping implementation}
The class \texttt{PingClient} has been implemented in the following file, \texttt{PingClient.java}:

\lstinputlisting[language=java]{PingClient.java}

A socket is opened in the \texttt{PingClient.run()} method in line 43. In line
44, we set the \texttt{receive()} timeout to 1000 milliseconds.

A ping request packet is formed in lines 53-55 and it is sent in line 59. Then,
we prepare an object in which the ping response is stored (lines 64-65), and we
wait for a server response in lines 66-74. If the server doesn't respond,
\texttt{receive()} wakes up and throws a \texttt{SocketTimeoutException}, which
we catch and handle in line 71. Otherwise, we calculate the time taken and
print out the size of the received packet in lines 68-69.

\subsection{Program output}

Launching a \texttt{PingServer} on the machine named \texttt{camel15}, port
8085, and launching a \texttt{PingClient} on the machine named
\texttt{camel08}, specifying destination hostname \texttt{camel15} and
destination port 8085, we get the following output:

\subsubsection{Server output}
\begin{lstlisting}[]
$ java PingServer 8085
Received from 130.225.16.127: PING 1 2010-12-12T15:03:03.756+0100
	Reply sent.
Received from 130.225.16.127: PING 2 2010-12-12T15:03:04.816+0100
	Reply sent.
Received from 130.225.16.127: PING 3 2010-12-12T15:03:05.871+0100
	Reply sent.
Received from 130.225.16.127: PING 4 2010-12-12T15:03:06.938+0100
	Reply sent.
Received from 130.225.16.127: PING 5 2010-12-12T15:03:08.055+0100
	Reply not sent.
\end{lstlisting}
\subsubsection{Client output}
\begin{lstlisting}[]
$ java PingClient camel15 8085
1024 bytes from camel15 (camel15/130.225.16.134): ping_seq=1 time=47960027ns
1024 bytes from camel15 (camel15/130.225.16.134): ping_seq=2 time=53758746ns
1024 bytes from camel15 (camel15/130.225.16.134): ping_seq=3 time=63897344ns
1024 bytes from camel15 (camel15/130.225.16.134): ping_seq=4 time=114912574ns
Timeout from camel15 (camel15/130.225.16.134)
\end{lstlisting}
The first four packets from the client get the appropriate attention from the
server, and for the fifth packet, the server simulates packet loss by
disregarding the request.
\end{document}
