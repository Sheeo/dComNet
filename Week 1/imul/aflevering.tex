\documentclass[12pt,a4paper]{article}
\usepackage[utf8]{inputenc}
\usepackage[danish]{babel}
\usepackage{amsmath,amssymb,amsthm}
\usepackage{listings}

\setlength{\parindent}{0in}
\setlength{\parskip}{0.1in}

\newcommand{\imul}{\texttt{imul}}

\newtheorem{theorem}{Sætning}

\title{Computere og Netværk (Q2/2010)}
\author{Mathias Rav, 20103940 \\
		Michael Søndergaard, 20104223 \\
		DAT-3}
\date{Uge 1, 7.\ november 2010}

\begin{document}
\maketitle

\section{Multiplikation med \texttt{imul.c}}
Kørsel af programmet \texttt{imul.c} gav:

$\text{imul}(2,3) = 6$.

Ved kørsel med en negativ værdi for $x$ fås resultatet $0$.  $x$, den første
parameter i \imul, indgår nemlig i while-løkkens betingelse: \texttt{while ( x
> 0 )}. Hvis $x$ ikke er positiv, så vil løkken ikke blive udført, og
resultatet af funktionen vil være $0$, selvom $xy\ne0$.

En kørsel med $x=2$ og $y=-4$ gav:

$\text{imul}(2,-4) = -8$.

Programmet vil altid multiplicere korrekt når $x$ er positiv, da
addition/subtraktion er veldefineret for alle heltal, sådan som det sker i
kodelinjen \texttt{p = p + y}.  Hvis $y$ er negativ, så vil $p$ blive mindre.

Rent matematisk har funktionen \imul{} en state, der kan angives entydigt ved
tuplen $(x,y,p)$. Så længe $x>0$ vil kroppen af while-løkken ændre staten af
funktionen: $x$ bliver \'en mindre, og $y$ bliver lagt til $p$. Dette
fortsætter indtil $x$ er mindre end eller lig med $0$, hvor resultatet af
funktionen er $p$. Det kan beskrives ved en funktion såsom
\[\delta(x,y,p)=\left\{\begin{array}{cc}
	\delta(x-1,y,p+y) & \text{hvis }x>0 \\
	p & \text{hvis }x\leq 0
\end{array}\right.\]

\begin{theorem}
	For $x,y,p\in\mathbb{Z},x\geq0$ er $\delta(x,y,p)=xy+p$.
\end{theorem}

\begin{proof}
	Vi beviser sætningen ved induktion på $x$.
	\paragraph{Basis $x=0$} Ved definitionen af $\delta$ gælder netop $\delta(0,y,p)=p=0\cdot y+p$.
	\paragraph{Induktion} Antag, at $\delta(x-1,y,p)=(x-1)y+p$. Så er
	\begin{align*}
		\delta(x,y,p)&=\delta(x-1,y,p+y) \\
		&=(x-1)y+p+y \\
		&=xy+p. \qedhere{}
	\end{align*}
\end{proof}

Af induktionsbeviset fremgår det, at $\delta(x,y,p)=xy+p$. Da variablen $p$ er
lokal i funktionen og initialiseret til $0$, må resultatet ved kørsel af
C-funktionen $\texttt{imul(int x, int y)}$ for positiv $x$ og vilkårlig $y$
være $\delta(x,y,0)=xy$.
\clearpage
\section{Multiplikation med \imul{}-programmet, i IJVM}
En kørsel af programmet imul.j ser således ud:

IJVM Trace of gen/imul.bc Sun Nov  7 15:08:28 2010

\begin{tabular}{llllllllll}
                    &                    & stack = 0,& 1,& 16 \\
bipush 44           &\texttt{[10 2c]    }& stack = 44,& 0,& 1,& 16 \\
bipush 2            &\texttt{[10 02]    }& stack = 2,& 44,& 0,& 1,& 16 \\
bipush 3            &\texttt{[10 03]    }& stack = 3,& 2,& 44,& 0,& 1,& 16 \\
invokevirtual 0     &\texttt{[b6 00 00] }& stack = 15,& 51,& 0,& 3,& 2,& 22,& 0,& 1 \\
bipush 0            &\texttt{[10 00]    }& stack = 0,& 15,& 51,& 0,& 3,& 2,& 22,& 0 \\
istore 3            &\texttt{[36 03]    }& stack = 15,& 51,& 0,& 3,& 2,& 22,& 0,& 1 \\
iload 1             &\texttt{[15 01]    }& stack = 2,& 15,& 51,& 0,& 3,& 2,& 22,& 0 \\
iflt 25             &\texttt{[9b 00 19] }& stack = 15,& 51,& 0,& 3,& 2,& 22,& 0,& 1 \\
iload 1             &\texttt{[15 01]    }& stack = 2,& 15,& 51,& 0,& 3,& 2,& 22,& 0 \\
ifeq 20             &\texttt{[99 00 14] }& stack = 15,& 51,& 0,& 3,& 2,& 22,& 0,& 1 \\
iload 1             &\texttt{[15 01]    }& stack = 2,& 15,& 51,& 0,& 3,& 2,& 22,& 0 \\
bipush 1            &\texttt{[10 01]    }& stack = 1,& 2,& 15,& 51,& 0,& 3,& 2,& 22 \\
isub                &\texttt{[64]       }& stack = 1,& 15,& 51,& 0,& 3,& 2,& 22,& 0 \\
istore 1            &\texttt{[36 01]    }& stack = 15,& 51,& 0,& 3,& 1,& 22,& 0,& 1 \\
iload 3             &\texttt{[15 03]    }& stack = 0,& 15,& 51,& 0,& 3,& 1,& 22,& 0 \\
iload 2             &\texttt{[15 02]    }& stack = 3,& 0,& 15,& 51,& 0,& 3,& 1,& 22 \\
iadd                &\texttt{[60]       }& stack = 3,& 15,& 51,& 0,& 3,& 1,& 22,& 0 \\
istore 3            &\texttt{[36 03]    }& stack = 15,& 51,& 3,& 3,& 1,& 22,& 0,& 1 \\
goto -24            &\texttt{[a7 ff e8] }& stack = 15,& 51,& 3,& 3,& 1,& 22,& 0,& 1 \\
iload 1             &\texttt{[15 01]    }& stack = 1,& 15,& 51,& 3,& 3,& 1,& 22,& 0 \\
iflt 25             &\texttt{[9b 00 19] }& stack = 15,& 51,& 3,& 3,& 1,& 22,& 0,& 1 \\
iload 1             &\texttt{[15 01]    }& stack = 1,& 15,& 51,& 3,& 3,& 1,& 22,& 0 \\
ifeq 20             &\texttt{[99 00 14] }& stack = 15,& 51,& 3,& 3,& 1,& 22,& 0,& 1 \\
iload 1             &\texttt{[15 01]    }& stack = 1,& 15,& 51,& 3,& 3,& 1,& 22,& 0 \\
bipush 1            &\texttt{[10 01]    }& stack = 1,& 1,& 15,& 51,& 3,& 3,& 1,& 22 \\
isub                &\texttt{[64]       }& stack = 0,& 15,& 51,& 3,& 3,& 1,& 22,& 0 \\
istore 1            &\texttt{[36 01]    }& stack = 15,& 51,& 3,& 3,& 0,& 22,& 0,& 1 \\
iload 3             &\texttt{[15 03]    }& stack = 3,& 15,& 51,& 3,& 3,& 0,& 22,& 0 \\
iload 2             &\texttt{[15 02]    }& stack = 3,& 3,& 15,& 51,& 3,& 3,& 0,& 22 \\
iadd                &\texttt{[60]       }& stack = 6,& 15,& 51,& 3,& 3,& 0,& 22,& 0 \\
istore 3            &\texttt{[36 03]    }& stack = 15,& 51,& 6,& 3,& 0,& 22,& 0,& 1 \\
\end{tabular}
\clearpage
\begin{tabular}{llllllllll}
goto -24            &\texttt{[a7 ff e8] }& stack = 15,& 51,& 6,& 3,& 0,& 22,& 0,& 1 \\
iload 1             &\texttt{[15 01]    }& stack = 0,& 15,& 51,& 6,& 3,& 0,& 22,& 0 \\
iflt 25             &\texttt{[9b 00 19] }& stack = 15,& 51,& 6,& 3,& 0,& 22,& 0,& 1 \\
iload 1             &\texttt{[15 01]    }& stack = 0,& 15,& 51,& 6,& 3,& 0,& 22,& 0 \\
ifeq 20             &\texttt{[99 00 14] }& stack = 15,& 51,& 6,& 3,& 0,& 22,& 0,& 1 \\
iload 3             &\texttt{[15 03]    }& stack = 6,& 15,& 51,& 6,& 3,& 0,& 22,& 0 \\
ireturn             &\texttt{[ac]       }& stack = 6,& 0,& 1,& 16 \\
ireturn             &\texttt{[ac]       }& stack = 6
\end{tabular}

return value: 6

Trace-outputtet ovenfor, viser kun de øverste 8 elementer af stakken. Dette er
vildledende, da man ved optælling får den maksimale stakstørrelse til at være 11.
Denne størrelse optræder i while-løkken, når vi foretager aritmetik: Første
gang når vi skal udregne $x-1$, og anden gang når vi skal udregne $p+y$.

Stakkens maksimale størrelse er den samme uanset input. Det gør ingen forskel,
hvor mange gange kroppen af løkken udføres, da denne så at sige rydder op efter
sig selv i stakken.

Efter ordren \texttt{invokevirtual imul} ser stakken således ud:

$\text{stack} = 15, 51, 0, 3, 2, 22, 0, 1$

Når \texttt{invokevirtual imul} køres, ser maskinen, at \imul{} tager to
argumenter. Så popper den to argumenter (2 og 3) samt en objektreference (44 -
denne smides væk) fra stakken og laver en ny stakbase med en link pointer (22),
parameterliste (2, 3), lokal variabelliste (0) samt den tidligere programtæller
(51) og link pointer-adresse (15).  Link pointeren (22), peger på
hukommelsescellen der indeholder den tidligere programtæller (51).  Tæller vi
nedad fra adresse nummer $22$ ved den tidligere programtæller finder vi
følgende adresser:

\begin{tabular}{|r|r|l|}
	\hline
	Adr. & Værdi & Forklaring \\
	\hline
	23 & 15 & tidligere LV \\
	22 & 51 & tidligere PC \\
	21 & 0 & lokal variabel \\
	20 & 3 & parameter \\
	19 & 2 & parameter \\
	18 & 22 & LV \\
	\hline
	17 & 0 & tidligere LV \\
	16 & 1 & tidligere PC \\
	15 & 16 & LV \\
	\hline
\end{tabular}

Hvorfor starter \texttt{main}s stak på adressen 15, og ikke på adressen 0?
Forklaringen her er, at de første 13 words (52 bytes) indeholder metodeområdet
og de næste 2 words indeholder programmets konstanter. Den tidligere
programtæller, 51, peger på den sidste instruktion i programmet, som
efterfølger kaldet til \texttt{invokevirtual} i \texttt{main}: \texttt{ac}
(\texttt{ireturn}).

\subsection{Antal ordre udført}
Antallet af ordre udført er konstant med henhold til $y$. Der bliver ikke lavet
nogen sammenligninger, der involverer $y$ - den bliver bare lagt til en
midlertidig variabel. Derimod afhænger antallet af ordre udført af $x$.

For hver gang $x$ bliver en større, skal der udføres 13 ordrer. Dette
svarer til det antal ordrer, som findes i kroppen af while-løkken.
Derudover skal der udføres 14 ordrer uden for løkken.
Antallet af ordrer $i$ som funktion af parameteren $x$ kan derfor
skrives som

\[i(x)=\left\{\begin{array}{cc}
	13x + 13 & \text{hvis }x\geq0 \\
	11 & \text{hvis }x<0
\end{array}\right.\]


For at eftervise denne model laver vi en tabel over antallet af ordrer
ved kørsel af \imul{} med forskellige parametre:

\begin{tabular}{|c|c|c|}
\hline
$x$ & $y$ & ordrer \\ \hline
-5  & 2  &  11 \\ \hline
-1  & 2  &  11 \\ \hline
 0  & 2  &  13 \\ \hline
 1  & 1  &  26 \\ \hline
 2  & 3  &  39 \\ \hline
 3  & 5  &  52 \\ \hline
 4  & 4  &  65 \\ \hline
50  & 1  & 663 \\ \hline
\end{tabular}

Disse data stemmer netop overens med vores model.

\subsection{Optimering af \imul{}}

En lineær sammenhæng mellem antallet af ordrer udført og parameteren til
metoden er skidt. Hvis vi erstatter instruktionen \texttt{bipush}, som er
begrænset til signed 8-bit integers i intervallet -128 til 127, med
instruktionen \texttt{ldc\_w}, kan vi køre metoden med hvilke som helst
parametre mellem $-2^{31}$ og $2^{31}-1$. For store værdier af $x$ tager dette
dog meget lang tid.

Vi udnytter, at $x\cdot y=\frac 12x\cdot2y$. Hvis vi kan lave denne omskrivning
på runtime i \imul{}, vil det gøre metoden hurtigere for større tal, da
while-løkken kun skal udføres halvt så mange gange (eller færre, hvis vi kan
genbruge trikket og omskrive til $\frac14x\cdot4y$, osv.).

Der findes dog ikke instruktioner til hverken at dividere eller gange i dette
instruktionssæt. (Hvis der gjorde, ville det gøre denne opgave betydeligt mere
uinteressant!) Vi kan dog nemt gange med 2 ved at omskrive $y=2y$ til
$y=y+y$, eller i IJVM til $\texttt{iload y} \quad \texttt{dup} \quad
\texttt{iadd} \quad \texttt{istore y}$.

I stedet for at dividere $x$ med $2$ indfører vi en ny variabel, $step$, som
angiver, hvor meget vi skal trække fra $x$ i hver iteration. Det giver sig
selv, at vores løkke bliver kørt lige mange gange, hvis vi trækker $step$ fra
$x$ hver gang, som hvis vi trækker $1$ fra $x/step$. At dividere $x$ med 2
giver samme resultat som at fordoble $step$, såfremt $x$ er deleligt med
$step$. Til at starte med er $step=1$, men hvis $x$ er deleligt med $2$, så
bliver $step$ og $y$ fordoblet.

For at tjekke om $x$ er deleligt med $step$ udnytter vi, at $step$ altid er en
potens af $2$. Vi kan nemlig afgøre, om et heltal $x$ er deleligt med $2^n$ ved
at tjekke de nederste $n-1$ bits af den binære repræsentation af $x$: Hvis de
er $0$, så er $x$ delelig med $2^n$. Vi indfører derfor en variabel $mask$, som
er lig $step*2-1$. Hvis $x\ \&\ mask=0$ (ved operationen bitwise AND, kaldet
IAND i IJVM), så er $x$ delelig med $step$, og vi kan fordoble $step$ og $y$ og
lægge den nye $step$ til $mask$ uden at ændre det endelige resultat af \imul.

Rent konkret realiseres det ovenstående ved at omskrive while-løkken i \imul{} til det følgende:

\begin{lstlisting}
while:                 // while
   iload x
   iflt end_while
   iload x
   ifeq end_while      //    ( x > 0 ) {
   iload x
   iload mask
   iand
   ifeq incrsub        //    if (x & mask != 0) {
   goto post_incrsub
incrstep:
   iload step
   dup
   iadd
   dup
   istore step         //       step += step;
   iload mask
   ior
   istore mask         //       mask |= step;
   iload y
   dup
   iadd
   istore y            //       y += y;
post_incrsub:          //    }
   iload x
   iload step
   isub
   istore x            //    x = x - step;
   iload p
   iload y
   iadd
   istore p            //    p = p + y;
   goto while          // }
end_while:
\end{lstlisting}

$mask$ og $step$ skal naturligvis erklæres som lokale variabler.

For små værdier af $x$ vil dette naturligvis give et større antal operationer,
men der vil hurtigt være tid at spare. Uden denne optimering tager det 663
ordrer at foretage en udregning med $x=50$; med optimeringen tager denne bare
169 ordrer.
%
%\begin{tabular}{rr|rr|rr|rr}
% 1 & 35   & 26 & 140  & 51 & 187  & 76 & 216 \\
% 2 & 46   & 27 & 158  & 52 & 187  & 77 & 234 \\
% 3 & 64   & 28 & 158  & 53 & 205  & 78 & 180 \\
% 4 & 64   & 29 & 176  & 54 & 169  & 79 & 198 \\
% 5 & 82   & 30 & 133  & 55 & 187  & 80 & 198 \\
% 6 & 75   & 31 & 151  & 56 & 187  & 81 & 216 \\
% 7 & 93   & 32 & 151  & 57 & 205  & 82 & 198 \\
% 8 & 93   & 33 & 169  & 58 & 187  & 83 & 216 \\
% 9 & 111  & 34 & 151  & 59 & 205  & 84 & 216 \\
%10 & 93   & 35 & 169  & 60 & 205  & 85 & 234 \\
%11 & 111  & 36 & 169  & 61 & 223  & 86 & 198 \\
%12 & 111  & 37 & 187  & 62 & 162  & 87 & 216 \\
%13 & 129  & 38 & 151  & 63 & 180  & 88 & 216 \\
%14 & 104  & 39 & 169  & 64 & 180  & 89 & 234 \\
%15 & 122  & 40 & 169  & 65 & 198  & 90 & 216 \\
%16 & 122  & 41 & 187  & 66 & 180  & 91 & 234 \\
%17 & 140  & 42 & 169  & 67 & 198  & 92 & 234 \\
%18 & 122  & 43 & 187  & 68 & 198  & 93 & 252 \\
%19 & 140  & 44 & 187  & 69 & 216  & 94 & 180 \\
%20 & 140  & 45 & 205  & 70 & 180  & 95 & 198 \\
%21 & 158  & 46 & 151  & 71 & 198  & 96 & 198 \\
%22 & 122  & 47 & 169  & 72 & 198  & 97 & 216 \\
%23 & 140  & 48 & 169  & 73 & 216  & 98 & 198 \\
%24 & 140  & 49 & 187  & 74 & 198  & 99 & 216 \\
%25 & 158  & 50 & 169  & 75 & 216  &100 & 216 \\
%\end{tabular}

\end{document}
