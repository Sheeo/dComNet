\documentclass[12pt,a4paper]{article}
\usepackage[utf8]{inputenc}
\usepackage[danish]{babel}
\usepackage{amsmath,amssymb,amsthm}
\usepackage{listings}
\usepackage{color}
\lstloadlanguages{C,sh,JVMIS}
\setlength{\parindent}{0in}
\setlength{\parskip}{0.1in}

\newcommand{\imul}{\texttt{imul}}

\newtheorem{theorem}{Sætning}

\lstset{ %
basicstyle=\footnotesize,       % the size of the fonts that are used for the code
numbers=left,                   % where to put the line-numbers
numberstyle=\footnotesize,      % the size of the fonts that are used for the line-numbers
stepnumber=1,                   % the step between two line-numbers. If it's 1 each line 
                                % will be numbered
numbersep=5pt,                  % how far the line-numbers are from the code
backgroundcolor=\color{white},  % choose the background color. You must add \usepackage{color}
showspaces=false,               % show spaces adding particular underscores
showstringspaces=false,         % underline spaces within strings
showtabs=false,                 % show tabs within strings adding particular underscores
tabsize=2,	          	        % sets default tabsize to 2 spaces
captionpos=b,                   % sets the caption-position to bottom
breaklines=true,                % sets automatic line breaking
breakatwhitespace=false,        % sets if automatic breaks should only happen at whitespace
title=\lstname,                 % show the filename of files included with \lstinputlisting;
                                % also try caption instead of title
}


\title{Computere og Netværk (Q2/2010)}
\author{Mathias Rav, 20103940 \\
		Michael Søndergaard, 20104223 \\
		DAT-3}
\date{Week 2, 14.\ november 2010}

\begin{document}
\maketitle

\section{Corrected version of \imul{} in C}

The following C function \texttt{imul.c} can do integer multiplication for
positive as well as negative arguments:

\lstset{language=C}
\lstinputlisting{./imul/imul.c.presentation}

\section{Corrected \imul{} in IJVM}

A corresponding IJVM method could look like this:

\lstset{language=JVMIS}
\lstinputlisting{./imul/imul.j.presentation}

A main program accepting two arguments has been crafted so the \texttt{imul}
method can be tested. The main program looks like this:

\begin{lstlisting}
.method main
.args 3
bipush 120 // objref
iload 1
iload 2
invokevirtual imul
ireturn
\end{lstlisting}

Running the main program with all four combinations of positive/negative
arguments gave the following output:

\lstset{language=sh,numbers=none}
\begin{lstlisting}
% ijvm-asm imul.j > imul.bc
% ijvm imul.bc 4 4
16
% ijvm imul.bc -5 9
-45
% ijvm imul.bc 3 -8
-24
% ijvm imul.bc -6 -7
42
\end{lstlisting}
\section{Improved \texttt{idiv}}

The C function \texttt{idiv} has been fixed and included in a C program. It has
been tested with all combinations of signs for $x$ and $y$. It looks like this:

\lstset{language=C,numbers=left}
\lstinputlisting{./idiv/idiv.c}

We have written a corresponding method \texttt{idiv} in IJVM. It looks like
this:

\lstset{language=JVMIS}
\lstinputlisting{./idiv/idiv.j.presentation}

It has undergone the same testing as the fixed C function.

\lstset{language=sh,numbers=none}
\begin{lstlisting}
% ijvm -s idiv.bc 10 3
return value: 3
% ijvm -s idiv.bc -42 2
return value: -21
% ijvm -s idiv.bc 57 -5
return value: -11
% ijvm -s idiv.bc -13 -4
return value: 3
\end{lstlisting}

\section{Udtryk fra SCO, side 368}
We computed the expression from SCO, page 368, with the following IJVM-program:
\lstset{language=JVMIS,numbers=left}
\lstinputlisting{./polish-arithmetic/polish.j.presentation}

The result was consistent with the result from the book:
\lstset{language=sh,numbers=none}
\begin{lstlisting}
% ijvm -s polish.bc
return value: 6
\end{lstlisting}

Using the special IJVM builtins \texttt{putchar} and \texttt{getchar}, we could
write a parser to interpret expressions in reverse polish notation from
standard input.

This notation is easy to parse with a stack-based machine such as the IJVM. An
integer is taken to be pushed onto the stack, and an operator pops two operands
and pushes the result of the operation.

\end{document}
