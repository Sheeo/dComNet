\documentclass[12pt,a4paper]{article}
\usepackage[utf8]{inputenc}
\usepackage[danish]{babel}
\usepackage{amsmath,amssymb,amsthm}
\usepackage{listings}
\usepackage{color}
\lstloadlanguages{C,sh,JVMIS}
\setlength{\parindent}{0in}
\setlength{\parskip}{0.1in}

\newcommand{\ishl}{\texttt{ishl}}
\newcommand{\ishr}{\texttt{ishr}}
\newcommand{\iushr}{\texttt{iushr}}
\newcommand{\ijvm}{\textsc{ijvm}}
\newcommand{\ijvmmal}{\texttt{ijvm.mal}}
\newcommand{\ijvmspec}{\texttt{ijvm.spec}}

\newtheorem{theorem}{Sætning}

\lstset{ %
language=java,
basicstyle=\footnotesize,       % the size of the fonts that are used for the code
numbers=left,                   % where to put the line-numbers
numberstyle=\footnotesize,      % the size of the fonts that are used for the line-numbers
stepnumber=1,                   % the step between two line-numbers. If it's 1 each line 
                                % will be numbered
numbersep=5pt,                  % how far the line-numbers are from the code
backgroundcolor=\color{white},  % choose the background color. You must add \usepackage{color}
showspaces=false,               % show spaces adding particular underscores
showstringspaces=false,         % underline spaces within strings
showtabs=false,                 % show tabs within strings adding particular underscores
tabsize=2,	          	        % sets default tabsize to 2 spaces
captionpos=b,                   % sets the caption-position to bottom
breaklines=true,                % sets automatic line breaking
breakatwhitespace=false,        % sets if automatic breaks should only happen at whitespace
title=\lstname,                 % show the filename of files included with \lstinputlisting;
                                % also try caption instead of title
}


\title{Computers and Networks (Q2/2010)}
\author{Mathias Rav, 20103940 \\
		Michael Søndergaard, 20104223 \\
		DAT-3}
\date{Week 4, November 28, 2010}

\begin{document}
\maketitle

\section{Extension of \ijvmmal}
The microprogram that implements \ijvm{} has been changed to add the opcodes \ishl, \ishr{} and \iushr. We have added the following microcode to \ijvmmal:
\begin{lstlisting}
ishl = 0x78:         # ishl: a << b
	MAR = SP = SP - 1; rd
	OPC = 1 << 8       # 256
	OPC = OPC >> 1     # 128
	OPC = OPC >> 1     # 64
	OPC = OPC >> 1     # 32
	OPC = H = OPC - 1  # 31
	OPC = H AND TOS    # 0x3F & b
	H = MDR            # a
ishl_while:
	H = MDR = MDR + H  # a <<= 1
	OPC = OPC - 1; if (Z) goto ishl_endwhile; else goto ishl_while  # --b || break
ishl_endwhile:
	TOS = MDR; wr      # return a
	goto main

ishr = 0x7A:         # a >> b, arithmetic
	MAR = SP = SP - 1; rd
	OPC = 1 << 8       # 256
	OPC = OPC >> 1     # 128
	OPC = OPC >> 1     # 64
	OPC = OPC >> 1     # 32
	H = OPC - 1  # 31
	TOS = H AND TOS    # 0x3F & b
ishr_while:
	MDR = MDR >> 1
	TOS = TOS - 1; if (Z) goto ishr_endwhile; else goto ishr_while # --b || break
ishr_endwhile:
	TOS = MDR; wr	
	goto main

iushr = 0x7C:				# iushr: a >> b, logical
	MAR = SP = SP - 1; rd
	OPC = 1 << 8       # 256
	OPC = OPC >> 1     # 128
	OPC = OPC >> 1     # 64
	OPC = OPC >> 1     # 32
	H = OPC - 1  # 31
	TOS = H AND TOS    # 0x3F & b
iushr_signbit:
	OPC = 1 << 8
	OPC = OPC >> 1
	OPC = OPC << 8
	OPC = OPC << 8
	OPC = OPC << 8
	H = OPC - 1
iushr_while:
	MDR = MDR >> 1
	MDR = MDR AND H
	TOS = TOS - 1; if (Z) goto iushr_endwhile; else goto iushr_while # --b || break
iushr_endwhile:
	TOS = MDR; wr	
	goto main
\end{lstlisting}
We have added the following to the specification file \ijvmspec:
\begin{lstlisting}
0x78 ishl
0x7A iushr
0x7C ishr
\end{lstlisting}
Finally, we have tested the opcodes using the following test program:
\begin{lstlisting}
.method main
.args 1

//------------------------
// test normal circumstances

// ishl(4,5) = 4 << 5 = 2^7 = 128
bipush 4
bipush 5
ishl
pop

// iushr(120,3) = 120 >> 3 = 15
bipush 120
bipush 3
iushr
pop

// iushr(-4,1) = 0b1111...1100 >> 1 = 0b0111...1110 = 2
bipush -4
bipush 1
iushr
pop

// ishr(-4,1) = -4 >> 1 = -2
bipush -4
bipush 1
ishr
pop

// ishr(5,2) = 5 >> 2 = 1
bipush 5
bipush 2
ishr      // 5 >> 2 = 0b101 >> 2 = 0b001 = 1
pop

//------------------------
// test edge cases

// ishl(55,99) = 55 >> (99 & 31) = 55 >> 3 = 0b00110111 >> 3 = 0b00000110 = 6
bipush 55
bipush 99
ishr
pop

// ishl and overflow
// operand: 0b1111 1111 0000 1111 1111 0000 0000 0000
ldc_w -0x00F01000 // 0xFF0FF000
bipush 42
// result (trunc): 0b(11 1111 1100) 0011 1111 1100 0000 0000 0000 0000 0000
// = 0x3FC00000
ishl
ireturn
\end{lstlisting}
Compiling \ijvmmal{} using \texttt{mic1-asm} and running with \texttt{mic1}
yields the following trace:
\begin{lstlisting}
Mic1 Trace of ijvm.mic1 with test.bc Sun Nov 28 23:23:01 2010

                                stack = 0, 1, 15
bipush 4            [10 04]     stack = 4, 0, 1, 15
bipush 5            [10 05]     stack = 5, 4, 0, 1, 15
ishl                [78]        stack = 128, 0, 1, 15
pop                 [57]        stack = 0, 1, 15
bipush 120          [10 78]     stack = 120, 0, 1, 15
bipush 3            [10 03]     stack = 3, 120, 0, 1, 15
iushr               [7c]        stack = 15, 0, 1, 15
pop                 [57]        stack = 0, 1, 15
bipush -4           [10 fc]     stack = -4, 0, 1, 15
bipush 1            [10 01]     stack = 1, -4, 0, 1, 15
iushr               [7c]        stack = 2147483646, 0, 1, 15
pop                 [57]        stack = 0, 1, 15
bipush -4           [10 fc]     stack = -4, 0, 1, 15
bipush 1            [10 01]     stack = 1, -4, 0, 1, 15
ishr                [7a]        stack = -2, 0, 1, 15
pop                 [57]        stack = 0, 1, 15
bipush 5            [10 05]     stack = 5, 0, 1, 15
bipush 2            [10 02]     stack = 2, 5, 0, 1, 15
ishr                [7a]        stack = 1, 0, 1, 15
pop                 [57]        stack = 0, 1, 15
bipush 55           [10 37]     stack = 55, 0, 1, 15
bipush 99           [10 63]     stack = 99, 55, 0, 1, 15
ishr                [7a]        stack = 6, 0, 1, 15
pop                 [57]        stack = 0, 1, 15
ldc_w 1             [13 00 01]  stack = -15732736, 0, 1, 15
bipush 42           [10 2a]     stack = 42, -15732736, 0, 1, 15
ishl                [78]        stack = 1069547520, 0, 1, 15
ireturn             [ac]        stack = 1069547520
return value: 1069547520
\end{lstlisting}
Notice the top of stack right after the calls to opcodes \ishl, \iushr{} and
\ishr: They are $128$, $15$, $2^{32}-2$, $-2$, $1$, $6$ and \texttt{0x3FC00000} just as expected.
\end{document}
