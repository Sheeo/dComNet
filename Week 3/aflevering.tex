\documentclass[12pt,a4paper]{article}
\usepackage[utf8]{inputenc}
\usepackage[danish]{babel}
\usepackage{amsmath,amssymb,amsthm}
\usepackage{listings}
\usepackage{color}
\lstloadlanguages{C,sh,JVMIS}
\setlength{\parindent}{0in}
\setlength{\parskip}{0.1in}

\newcommand{\isub}{\texttt{isub}}
\newcommand{\imul}{\texttt{imul}}

\newtheorem{theorem}{Sætning}

\lstset{ %
basicstyle=\footnotesize,       % the size of the fonts that are used for the code
numbers=left,                   % where to put the line-numbers
numberstyle=\footnotesize,      % the size of the fonts that are used for the line-numbers
stepnumber=1,                   % the step between two line-numbers. If it's 1 each line 
                                % will be numbered
numbersep=5pt,                  % how far the line-numbers are from the code
backgroundcolor=\color{white},  % choose the background color. You must add \usepackage{color}
showspaces=false,               % show spaces adding particular underscores
showstringspaces=false,         % underline spaces within strings
showtabs=false,                 % show tabs within strings adding particular underscores
tabsize=2,	          	        % sets default tabsize to 2 spaces
captionpos=b,                   % sets the caption-position to bottom
breaklines=true,                % sets automatic line breaking
breakatwhitespace=false,        % sets if automatic breaks should only happen at whitespace
title=\lstname,                 % show the filename of files included with \lstinputlisting;
                                % also try caption instead of title
}


\title{Computers and Networks (Q2/2010)}
\author{Mathias Rav, 20103940 \\
		Michael Søndergaard, 20104223 \\
		DAT-3}
\date{Week 3, November 21, 2010}

\begin{document}
\maketitle
\section{Domain of the integers}
The IJVM machine represents signed integers using 32-bit two's complement.
Thus, the machine is able to represent the signed integers from $-2^{31}$ (the
bit string $100\dots000$) to $2^{31}-1$ (the bit string $011\dots111$).
\section{\texttt{geq.j}}
The program \texttt{geq.j} implements the relation $a\geq b$ using integer
subtraction in the function $geq:B_{32}\times B_{32}\rightarrow \{0,1\}$. This
may lead to overflow when comparing integers with differing signs.

For instance, when $a=100\dots000, |a|_2=-2^{31}$ and $|b|_2$ is positive,
i.e.\ with the two's complement representation that has a zero in the most
significant bit, the difference $\isub(a,b)$ will overflow and yield a positive
integer, $|\isub(a,b)|_2=2^{32}+|a|_2-|b|_2$. Rather than a negative
difference, which would indicate that $|a|_2<|b|_2$, the code will believe that
$|a|_2\geq |b|_2$ and return $1$.

\subsection{Overflow example: \texttt{geq}$(-2^{31}, 42)$}
\begin{tabular}{rl}
	$a=10000000 \dots 00000000$ & $|a|_2=-2^{31}$ \\
	$b=00000000 \dots 00101010$ & $|b|_2=42$ \\
	$c=\isub(a,b)=01111111 \dots 11010110$ & $|c|_2=2147483606=2^{31}-42$
\end{tabular}

As the value pushed by the \texttt{isub} operation is interpreted as a positive
integer, even though $|a|_2$ is negative and $|b|_2$ is positive and so
$|a|_2-|b|_2$ is negative, the \texttt{geq} method returns $1$, meaning
$|a|_2\geq |b|_2$.

\subsection{Example runs}

\begin{tabular}{rrcl}
	$a$ & $b$ & $a\geq b$? & $\texttt{geq}(a, b)$ \\
	\hline
	$2$ & $5$ & No & return value: $0$ \\
	$5$ & $2$ & Yes & return value: $1$ \\
	$-10$ & $10$ & No & return value: $0$ \\
	$12$ & $-8$ & Yes & return value: $1$ \\
	$-2000000000$ & $1000000000$ & No & return value: $1$ \\
	$2000000000$ & $-1000000000$ & Yes & return value: $0$ \\
	$-2000000000$ & $-1000000000$ & No & return value: $0$ \\
	$-1000000000$ & $-2000000000$ & Yes & return value: $1$
\end{tabular}

For $|a|\ll2^{31}$ and $|b|\ll2^{31}$, and when the signs of $a$ and $b$ are
the same, this $geq$ implementation works just fine. However, in the fifth and
sixth runs, the return value of $geq$ does not correspond with $a\geq b$ due to
overflow in the subtraction.

\section{Fixing \texttt{geq}}

We may avoid overflow in subtraction by checking the signs of the operands
before subtracting. If $|a|_2$ is positive and $|b|_2$ is negative, then
$|a|_2\geq |b|_2$ for any $a$ and $b$. Likewise, if $|a|_2$ is negative and
$|b|_2$ positive, then $|a|_2<|b|_2$ for any $a$ and $b$. Luckily, this is easy
to test, even in IJVM.

First, in C:

\lstset{language=C}
\begin{lstlisting}
int geq(int a, int b) {
	if (a < 0) {
		if (b < 0) {
			// a and b are both negative
			if (a-b < 0) {
				return 0;
			} else {
				return 1;
			}
		} else {
			// a negative, b positive
			return 0;
		}
	} else {
		if (b < 0) {
			// a positive, b negative
		} else {
			// a and b are both positive
			if (a-b < 0) {
				return 0;
			} else {
				return 1;
			}
		}
	}
}
\end{lstlisting}
This code is easy to translate into IJVM.
\lstset{language=JVMIS}
\begin{lstlisting}
.method geq
.args 3
	iload 1
	iflt ANEG
APOS:
	iload 2
	iflt APOSBNEG
	goto SAMESIGN // a positive, b positive
ANEG:
	iload 2
	iflt SAMESIGN // a negative, b negative
	goto ANEGBPOS
APOSBNEG:
	bipush 1
	ireturn
ANEGBPOS:
	bipush 0
	ireturn
SAMESIGN:
	iload 1
	iload 2
	isub
	iflt ELSE
THEN:
	bipush 1
	ireturn
ELSE:
	bipush 0
	ireturn
\end{lstlisting}

\subsection{Example runs}

\begin{tabular}{rrcl}
	$a$ & $b$ & $a\geq b$? & $\texttt{geq}(a, b)$ \\
	\hline
	$2$ & $5$ & No & return value: $0$ \\
	$5$ & $2$ & Yes & return value: $1$ \\
	$-10$ & $10$ & No & return value: $0$ \\
	$12$ & $-8$ & Yes & return value: $1$ \\
	$-2000000000$ & $1000000000$ & No & return value: $0$ \\
	$2000000000$ & $-1000000000$ & Yes & return value: $1$ \\
	$-2000000000$ & $-1000000000$ & No & return value: $0$ \\
	$-1000000000$ & $-2000000000$ & Yes & return value: $1$
\end{tabular}

Run with the same test cases as before, we see that the return values of $geq$
in the fifth and sixth table entry are now correct when compared with $a\geq
b$.
\section{$a=b$}
In IJVM, to branch if two values are equal, you push them to the stack,
subtract them using \texttt{ISUB}, and branch if the result is equal to zero
using \texttt{IFEQ}.

In this case, overflow doesn't matter, since in two's complement arithmetic, if
$a\ne b$, then $|isub(a, b)|_2\ne 0$, regardless of overflow.

Likewise, if $a=b$, then $|isub(a, b)|_2=0$. Therefore, when using subtraction
to check if $a=b$, \texttt{IFEQ} will branch if and only if $a=b$.
\end{document}
